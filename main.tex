\documentclass{article}
\usepackage{graphicx}

\begin{document}
\title{Intr to }


\author{Author's Name}

\maketitle

\begin{abstract}
    Variations in the Earth's orbit govern the insolation and its climate. The astronomical signals, which have been recovered in geological records , revolutionized the accuracy and precision of determination of the geological time scale 
    The orbital variations beyond 50 Myr cannot be reliably predicted because of the chaotic dynamics of the Solar System 
    However, the chaotic orbital evolution in the past could be constrained by geological data: Among 11 possible different astronomical evolutions, one was found to match the Newark-Hartford (NH) data from 210 Myrs ago 
    With 10,000 numerical integrations of 300 Myr from averaged equations of simplified Solar System, we have found several solutions that match better with the NH data, and a robust mechanism of the transition in Libsack record.
    Statistical analysis was also performed to obtain the probability density functions (PDF) of fundamental frequencies of the Solar System, that could be roughly approximated by Gaussian functions. The evolution of the PDF is found to be similar to the one of a diffusive process.


\end{abstract}



\section{Introduction}
Here is the text of your introduction.

\begin{equation}
    \label{simple_equation}
    \alpha = \sqrt{ \beta }
\end{equation}

\subsection{Subsection Heading Here}
Writttte your subsection text here.

\section{Conclusion}
Write your conclusion here.

\end{document}
